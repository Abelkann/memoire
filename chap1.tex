\chapter{Généralités sur les systèmes de mise à la terre}

	\section{Introduction}

		\section{Aspects conceptuels}

			\subsection{Définitions}
				Le système de mise requiert plusieurs termes qui sont rattachés à lui et qui méritent d’être clarifiés, notamment:
				\subsubsection{Mise à la terre}
					
		\section{Objet et constitution d'une mise à la terre}
			\subsection{Objet}
				Le rôle d'un système de mise à la terre dans une installation électrique est de permettre d'écouler à l'intérieur du sol de courants de toutes origines, qu'il s'agisse de courants de choc dus à des coups de foudre ou bien de courants de défauts d'une fréquence donnée.\cite{edf89}

Lors de l'écoulement de tels courants par une prise ou un réseau de terre, des différences de potentiel peuvent apparaître entre certains points, par exemple entre la prise de terre et le sol qui l'entoure, ou entre deux points du sol. La conception des prises et réseaux de terre doit permettre, même dans ces conditions, d'assurer le maintien de :
	\begin{itemize}
		\item la sécurité des personnes et des animaux;
		\item la protection des installations de puissance;
		\item la protection des équipements sensibles;
		\item garantir un potentiel de référence.
	\end{itemize}
	
				\subsubsection{La sécurité des personnes et des animaux}
					Lors de l'écoulement dans le sol de courants élevés, la sécurité doit être assurée à l'intérieur de l'installation électrique et de ses abords immédiats par une limitation de la tension de pas et de la tension de contact à des valeurs non dangereuses ou admissibles pour le corps humain ou les animaux. Cette limitation est obtenue grâce à la connaissance et au contrôle de la répartition du potentiel à la surface du sol.
					
				\subsubsection{Protection des installations de puissance}
					Le système de mise à la terre des installations électrique, la prise de terre des supports de lignes, limitent la création et la propagation des surtensions provoquées par les défauts à 50 ou 60 Hz, les manœuvres d'appareillages dans les postes et centrales sans pour autant omettre l'impact dangereux que présente la foudre.
					
				\subsubsection{Protection des équipements sensibles}
					A côté des installations de puissance, on trouve également et très souvent des équipements fonctionnant à des niveaux de tension comparativement beaucoup plus bas. parmi ces équipements nous pouvons citer les équipements de relayage dans les postes mais aussi des câbles P.T.T installés à proximité dans les lignes, des postes ou des centrales. Ces équipements sont en même temps exposés aux effets des surtensions subies par les installations de puissance avec lesquels ils peuvent être liés par un couplage quelconque.
					
				\subsubsection{Potentiel de référence}
					Différents équipements placés dans une installation doivent, lorsqu'ils sont reliés électriquement, rester fixés à un potentiel identique électrique même pendant la durée des perturbations mentionnées plus haut.\\							
Dans les réseaux à neutre directement à la terre, le réseau de terre des postes contribue à fixer le potentiel des phases saines pendant un défaut, mais les courants de défaut peuvent alors atteindre des valeurs importantes.\\
Ces quelques exemples montrent d'une manière particulière l'importance que revêt un système ou réseau de mise à la terre dans une installation électrique et sa qualité de fonctionnement.

			\subsection{Constitution d'une mise à la terre}
				La constitution d'une mise à la terre ne fait l'unanimité puisque celle-ci dépend de plusieurs paramètres, notamment :\\
				
				\begin{itemize}
					\begin{itemize}
					\item le niveau de tension, selon qu'il s'agit d'une installation basse, moyenne ou haute tension;
					\item le niveau de complexité du système ainsi construit;
					\item la finalité d'utilisation en référence aux bâtiments devant recevoir la mise à la terre;
					\item les spécifications techniques de l'installation.
					\end{itemize}
				\end{itemize}
\\				
Les systèmes de mise à la terre constituent en outre un moyen de dérivation efficace des courants de défaut, de coups de foudre, de décharges électrostatiques, des interférences électromagnétiques et des interférences haute fréquence. La mise à la terre se construit en établissant de manière volontaire une connexion entre la ligne d'alimentation(le neutre généralement) et une électrode placée dans le sol(prise de terre).\\

Une prise de terre est habituellement composée d'une ou plusieurs électrodes verticales(encore appelées piquets) ou horizontales(grilles). La forme de la prise de terre est normalement commandée par l'emplacement physique des appareils et des structures métalliques à connecter à la terre.\\

Les prises de mise à la terre se composent des trois(03) éléments, qui sont :\\

\begin{itemize}
\item le conducteur de terre;
\item l'électrode (piquet de terre, prise de terre profondément enfui dans le sol);
\item le contact entre le conducteur de terre et l'électrode;
\end{itemize}

\section{Tensions de sécurité}

\section{Plages des courants admissibles}

\section{Courant électrique à travers un corps humain}

Le corps humain conduit l'électricité. Même les faibles courants peuvent causer des effets de santé graves. Les spasmes, les brûlures, la paralysie de muscle, ou la mort peuvent résulter, selon la quantité du courant traversant le corps, l'itinéraire qu'il prend, et la durée de l'exposition.\\
