\chapter{Principales normes et critères de performances d'une mise à la terre}

	\section{Introduction}
	\section{Norme NF C15 - 100}
	\section{Norme IEC 61936 - 1}
	\section{Norme EN 50522}
	\section{Norme NF EN 62305 - 3}
	\section{Norme NF C 17 - 200}
	\section{Norme NF C 13 - 200}
	\section{Norme IEEE 80}
		\subsection{Présentation de la norme}
		\subsection{Paramètres de performances}
		\subsection{Étapes de conception d'une mise à la terre}
			\begin{description}
			\item[Détermination de la résistivité du sol]
			\item[Dimensionnement des conducteurs]
			\item[Détermination des tensions de contact et de pas admissibles]
			\item[Conception préliminaire ou initiale]
			\item[Estimation (détermination) de la résistance de la grille]
			\item[Courant maximum de la grille $I_{G}$]
			\item[Élévation du potentiel de terre (EPT)]
			\item[Calcul des tensions de maille et de pas]
				\begin{enumerate}
				\item Tension de maille (Em)
				\item Tension de pas (Es)
				\end{enumerate}
			\item[Possible révision]
			\item[Possible révision]
			\item[Amélioration de la configuration initiale]
			\item[Configuration finale ou conception détaillée]
			\end{description}
		\subsubsection{Principaux avantages et inconvénients de la norme}
	\section{Calcul des paramètres}
		\subsection{Choix d'un logiciel}
		\subsection{Présentation du logiciel}
		